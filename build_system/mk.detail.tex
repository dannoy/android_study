\documentclass[a4paper,11pt]{article}
\usepackage{CJK} %使用CJK套件
%\usepackage[encapsulated]{CJK} %1.Dec. 2009更新:使用[encapsulate]才是正確的用法
%\usepackage{CJK} %1.Dec. 2009更新:使用[encapsulate]才是正確的用法
\usepackage{indentfirst} %段首缩进

\author{Dannoy.Lee \\
    dannoy.lee@gmail.com}
% define the title
\title{Detail of Android Build System}
\begin{document}
%\begin{CJK}{UTF8}{bsmi} %開始CJK環境,設定編碼,設定字體
\begin{CJK*}{UTF8}{gbsn}
% generates the title
\maketitle
%\begin{center}
%dannoy.lee@gmail.com
%\end{center}

% insert the table of contents
\tableofcontents
\newpage

\section{Brief Introduction}
本文试图对Android的build system进行较详细的分析。但由于对Andrid平台环境不够熟悉,理解上可能会出现偏差。
\subsection{准备知识}
略。
\subsection{重要的配置文件}
\begin{enumerate}
    \item AndroidProducts.mk \\
    \item BoardConfig.mk \\
    \item CleanSpec.mk \\
\end{enumerate} 
\subsection{Two kinds of Target}
\begin{description}
    \item 正常的目标 \\
    \item 改变行为的目标(modifier targets) \\
\end{description} 
\ldots{}

\section{流程}
    \subsection{main.mk}
    \begin{itemize}
        \item include \$(BUILD\_SYSTEM)/help.mk \\
        \item include \$(BUILD\_SYSTEM)/config.mk \\
        \item include \$(BUILD\_SYSTEM)/cleanbuild.mk \\
        \begin{enumerate}
            \item include \$(BUILD\_SYSTEM)/cleanspec.mk:包含所有的CleanSpec.mk文件。
            \item 检查\$(PRODUCT\_OUT)/clean\_steps.mk里是否有非来自于CleanSpec.mk里的clean step,并重写该文件。
            \item 检查\$(PRODUCT\_OUT)/previous\_build\_config.mk里前一次的配置是否与这一次的配置是否一样,若不一样且DISABLE\_AUTO\_INSTALLCLEAN不等于true则进行必要的清理。
        \end{enumerate} 
        \item -include \$(OUT\_DIR)/versions\_checked.mk \\
            检查文件系统是否对大小写敏感,当前目录的绝对路径是否包含空格,java以及javac的版本是否正确(当前为1.6)。
        \item include \$(BUILD\_SYSTEM)/definitions.mk \\
        \begin{enumerate}
            \item 定义一堆ALL\_变量
            \item 定义一堆all-xx-files-under,all-subdir-xx-files,get-xxx,transform-xxx[-to-yyy],
            \item 
        \end{enumerate} 
    \end{itemize} 


    \subsection{config.mk}
    \begin{itemize}
        \item \-include \$(TOPDIR)buildspec.mk \\
        \item include \$(BUILD\_SYSTEM)/envsetup.mk
    \end{itemize} 

    \begin{enumerate} 
        \item 根据从envsetup.mk中得到的TARGET\_DEVICE来包含对应的BoardConfig.mk \newline
            131 board\_config\_mk := \textbackslash \\
            132     \$(strip \$(wildcard \textbackslash \\
            133         \$(SRC\_TARGET\_DIR)/board/\$(TARGET\_DEVICE)/BoardConfig.mk \textbackslash \\
            134         device/*/\$(TARGET\_DEVICE)/BoardConfig.mk \textbackslash \\
            135         vendor/*/\$(TARGET\_DEVICE)/BoardConfig.mk \textbackslash \\
            136     )) \\
        \item 通过include combo/select.mk设置HOST以及TARGET的编译器以及编译选项 \newline
        \begin{enumerate} 
            \item combo/select.mk设置通用的编译器,库选项,编译器选项以及ccache的封装\newline
            \item include \$(BUILD\_COMBOS)/\$(combo\_target)\$(combo\_os\_arch).mk 设置特定平台的编译器,编译器选项 \newline
        \end{enumerate} 
        \item 通过include \$(BUILD\_SYSTEM)/combo/javac.mk设置COMMON\_JAVAC, HOST\_JAVAC, TARGET\_JAVAC \\
        \item 定义LEX,YACC,DOXYGEN,AAPT,AIDL,PROTOC,ICUDATA,SIGNAPK\_JAR,MKBOOTFS等命令 \\
        \item 确立最终的CFLAGS,LDFALGS,CPPFLAGS等 \\
        \item include \$(BUILD\_SYSTEM)/dumpvar.mk \\
        \begin{enumerate} 
            \item 支持如下形式的用法,在build/envsetup.sh或者调试时用来获取makefile内部变量的值 \\ 
              CALLED\_FROM\_SETUP=true \textbackslash \\
                make -f config/envsetup.make dumpvar-TARGET\_OUT \\
              CALLED\_FROM\_SETUP=true \textbackslash \\
                make -f config/envsetup.make dumpvar-abs-HOST\_OUT\_EXECUTABLES \\
            \item 当dump的var不为report\_config时,打印出当前的makefile配置信息,即在每次编译时打印出的以下信息 \\
            ============================================ \\
            PLATFORM\_VERSION\_CODENAME=AOSP \\
            PLATFORM\_VERSION=4.0.4.0.4.0.4 \\
            TARGET\_PRODUCT=full \\
            TARGET\_BUILD\_VARIANT=eng \\
            TARGET\_BUILD\_TYPE=release \\
            TARGET\_BUILD\_APPS= \\
            TARGET\_ARCH=arm \\
            TARGET\_ARCH\_VARIANT=armv7-a \\
            HOST\_ARCH=x86 \\
            HOST\_OS=linux \\
            HOST\_OS\_EXTRA=Linux-3.0.0-8-generic-x86\_64-with-Ubuntu-10.04-lucid \\
            HOST\_BUILD\_TYPE=release \\
            BUILD\_ID=OPENMASTER \\
            OUT\_DIR=out \\
            =========================================== \\
        \end{enumerate} 
    \end{enumerate} 



    \subsection{envsetup.mk}
      根据系统设置一些HOST\_,BUILD\_之类的变量,再通过包含product\_config.mk确定所编译的product并设置HOST\_,TARGET\_等的用来指示输出目录变量。
    \subsection{product\_config.mk}
      这个里面要注意的就是一个product可以inherit另一个product的makefile的方式进行继承变量。

\section{Internals}
    \subsection{product}
    \subsection{clean step}
    \emph{主要文件:}
    \begin{description}
        \item cleanbuild.mk
        \item CleanSpec.mk
    \end{description}
    \subsection{building}
    \emph{主要文件:}
        Android.mk?




%\end{CJK} %有始有終
\end{CJK*} %有始有終
\end{document}

