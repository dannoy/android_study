\documentclass[a4paper,11pt]{article}
\usepackage{CJK} %使用CJK套件
\usepackage{indentfirst} %段首缩进
\usepackage{graphicx}
\usepackage{booktabs}
%\usepackage[T1]{fontenc}
\usepackage{xcolor}
\usepackage{listings}
\usepackage{float}

\author{Li.Jin \\
    dannoy.lee@gmail.com
    }
% define the title
\begin{document}
\begin{CJK*}{UTF8}{gbsn}
\title{Android SystemUI 之 Phone or Tablet}
\hbadness=10000

\tolerance=10000

\hfuzz=150pt
%\begin{CJK}{UTF8}{bsmi} %開始CJK環境,設定編碼,設定字體
%\pagestyle{headings}
%设置listings的全局属性(包含代码)
%\lstset{numbers=left, numberstyle=\tiny, keywordstyle=\color{blue!70}, commentstyle=\color{red!50!green!50!blue!50}, frame=shadowbox, rulesepcolor=\color{red!20!green!20!blue!20}}
\lstset{breaklines,numbers=left, showstringspaces=false, extendedchars=false, basicstyle=\ttfamily\small,numberstyle=\tiny, keywordstyle=\color{blue!70}, commentstyle=\color{red!50!green!50!blue!50}, frame=line, rulesepcolor=\color{red!20!green!20!blue!20}}
%\lstset{numbers=left, breaklines=true, numberstyle=\tiny, keywordstyle=\color{blue!70}, commentstyle=\color{red!50!green!50!blue!50}, frame=shadowbox, rulesepcolor=\color{red!20!green!20!blue!20}}
% generates the title
\maketitle

% insert the table of contents
%\tableofcontents
%\newpage

\setcounter{page}{1}
\section{引言}
本文分析了SystemUI启动过程中是如何确定显示PhoneStatusBar还是TabletStatusBar,以及
给出了指定PhoneStatusBar或者TabletStatusBar的方法。


\section{流程分析}
    \begin{enumerate}
        \item 在SystemServer(services/java/com/android/server/SystemServer.java)启动过程中有如下代码:
\begin{lstlisting}[language=JAVA]
ActivityManagerService.self().systemReady(new Runnable() {
        public void run() {
        Slog.i(TAG, "Making services ready");

        startSystemUi(contextF);                                                                                                                                                      
        try {
        if (batteryF != null) batteryF.systemReady();
        } catch (Throwable e) {
        reportWtf("making Battery Service ready", e); 
        }
         ...
}
\end{lstlisting}
        其中startSystemUi()启动了com.android.systemui.SystemUIService这个服务。
        \item SystemUIService的onCreate方法:
\begin{lstlisting}[language=JAVA]
public void onCreate() {
// Pick status bar or system bar.                                                                                                                                                     
if (SHOW_STATUS_BAR == getHideStartBarSetParam(this))
{   
    if (null == mServices)
    {   
        IWindowManager wm = IWindowManager.Stub.asInterface(
                ServiceManager.getService(Context.WINDOW_SERVICE));
        try {
            SERVICES[0] = wm.canStatusBarHide()
                ? R.string.config_statusBarComponent
                : R.string.config_systemBarComponent;
        } catch (RemoteException e) {
            Slog.w(TAG, "Failing checking whether status bar can hide", e); 
        }   
             ...
    }
}
\end{lstlisting}
       line 10根据wm.canStatusBarHide()决定接下来要启动的是哪一类型的statusbar,其中使用的
       资源文件内容如下(res/values/config.xml):
\begin{lstlisting}[language=XML]
<!-- Component to be used as the status bar service.  Must implement the IStatusBar
interface.  This name is in the ComponentName flattened format (package/class)  -->
<string name="config_statusBarComponent" translatable="false">com.android.systemui.statusbar.phone.PhoneStatusBar</string>                                                                

<!-- Component to be used as the system bar service.  Must implement the IStatusBar
interface.  This name is in the ComponentName flattened format (package/class)  -->
<string name="config_systemBarComponent" translatable="false">com.android.systemui.statusbar.tablet.TabletStatusBar</string>

\end{lstlisting}
        此外还需要说一下的是前一段onCreate里的第三行。
        这个是在原版android的基础之上添加的,其目的是为了支持"Settings -- Display ---Hide StatusBar"的设置的,
        通过该设置可以动态选择是否选择加载StatusBar。
        \item 继续分析canStatusBarHide()的调用过程,其最终会调用\\
        policy/src/com/android/internal/policy/impl/PhoneWindowManager.java
        里的该函数返回PhoneWindowManager.mStatusBarCanHide,其值由setInitialDisplaySize()里的如下代码初始化:
\begin{lstlisting}[language=JAVA]
if (width > height) {
    shortSize = height;
    ...
} else {
    shortSize = width;
    ...
}    

// Determine whether the status bar can hide based on the size
// of the screen.  We assume sizes > 600dp are tablets where we
// will use the system bar.
int shortSizeDp = shortSize
* DisplayMetrics.DENSITY_DEFAULT
/ DisplayMetrics.DENSITY_DEVICE;
mStatusBarCanHide = shortSizeDp < 600; 
\end{lstlisting}
    其中的width和height都是setInitialDisplaySize()传进来的参数。
    可以看出当屏幕的高和宽当中较小的那个换算为DPI后若小于600则mStatusBarCanHide值为true,
    此时为将加载PhoneStatusBar;反之,则加载TabletStatusBar。\\
    同时又有如下代码段:
\begin{lstlisting}[language=JAVA]
public class DisplayMetrics {
    public static final int DENSITY_LOW = 120;
    public static final int DENSITY_MEDIUM = 160;
    public static final int DENSITY_TV = 213;
    public static final int DENSITY_HIGH = 240;
    public static final int DENSITY_XHIGH = 320;
    public static final int DENSITY_DEFAULT = DENSITY_MEDIUM;
    public static final int DENSITY_DEVICE = getDeviceDensity();
    private static int getDeviceDensity() {                                                                                                                                                   
        // qemu.sf.lcd_density can be used to override ro.sf.lcd_density
        // when running in the emulator, allowing for dynamic configurations.
        // The reason for this is that ro.sf.lcd_density is write-once and is
        // set by the init process when it parses build.prop before anything else.
        return SystemProperties.getInt("qemu.sf.lcd_density",
                SystemProperties.getInt("ro.sf.lcd_density", DENSITY_DEFAULT));
    }

}
\end{lstlisting}
    当我们不在虚拟机上,且未设置ro.sf.lcd\_density属性,则DENSITY\_DEFAULT和\\
    DENSITY\_DEVICE
    相等,其值直接为宽和高分辨率的最小值。

    \item setInitialDisplaySize()由WindowManagerService.displayReady()调用,而后者
    又由SystemServer的启动代码直接调用(wm.displayReady())。
\begin{lstlisting}[language=JAVA]
public void displayReady() {
    synchronized(mWindowMap) {
        if (mDisplay != null) {
            throw new IllegalStateException("Display already initialized");
        }
        WindowManager wm = (WindowManager)mContext.getSystemService(Context.WINDOW_SERVICE);
        mDisplay = wm.getDefaultDisplay();
        synchronized(mDisplaySizeLock) {
            mInitialDisplayWidth = mDisplay.getRawWidth();
            mInitialDisplayHeight = mDisplay.getRawHeight();
            int rot = mDisplay.getRotation();
            if (rot == Surface.ROTATION_90 || rot == Surface.ROTATION_270) {
                // If the screen is currently rotated, we need to swap the
                // initial width and height to get the true natural values.
                int tmp = mInitialDisplayWidth;
                mInitialDisplayWidth = mInitialDisplayHeight;
                mInitialDisplayHeight = tmp;
            }
            mBaseDisplayWidth = mCurDisplayWidth = mAppDisplayWidth = mInitialDisplayWidth;
            mBaseDisplayHeight = mCurDisplayHeight = mAppDisplayHeight = mInitialDisplayHeight;
        }
        mInputManager.setDisplaySize(Display.DEFAULT_DISPLAY,
                mDisplay.getRawWidth(), mDisplay.getRawHeight(),
                mDisplay.getRawExternalWidth(), mDisplay.getRawExternalHeight());
        mPolicy.setInitialDisplaySize(mInitialDisplayWidth, mInitialDisplayHeight);                                                                                                       
    }
\end{lstlisting}
        其中的mDisplay.getRawWidth()与mDisplay.getRawHeight()实现如下:
\begin{lstlisting}[language=JAVA]
public int getRawWidth() {
    int w = getRawWidthNative();
    if (DEBUG_DISPLAY_SIZE) Slog.v(
            TAG, "Returning raw display width: " + w);                                                                                                                                    
    return w;
}   
private native int getRawWidthNative();

public int getRawHeight() {
    int h = getRawHeightNative();
    if (DEBUG_DISPLAY_SIZE) Slog.v(
            TAG, "Returning raw display height: " + h); 
    return h;
}   
private native int getRawHeightNative();

\end{lstlisting}
        可以看到实际上是调用了两个JNI的函数,接下来我们就开始到了C++代码。
        \item 代码在core/jni/android\_view\_Display.cpp,代码如下:
\begin{lstlisting}[language=C++]
static jint android_view_Display_getRawWidthNative(
        JNIEnv* env, jobject clazz)
{
    DisplayID dpy = env->GetIntField(clazz, offsets.display);
    return SurfaceComposerClient::getDisplayWidth(dpy);                                                                                                                                       
}

static jint android_view_Display_getRawHeightNative(
        JNIEnv* env, jobject clazz)
{
    DisplayID dpy = env->GetIntField(clazz, offsets.display);
    return SurfaceComposerClient::getDisplayHeight(dpy);
}
\end{lstlisting}
        其中2个函数内容如下(libs/gui/SurfaceComposerClient.cpp):
\begin{lstlisting}[language=C++]
ssize_t SurfaceComposerClient::getDisplayWidth(DisplayID dpy)
{
    if (uint32_t(dpy)>=NUM_DISPLAY_MAX)
        return BAD_VALUE;
    volatile surface_flinger_cblk_t const * cblk = get_cblk();
    volatile display_cblk_t const * dcblk = cblk->displays + dpy;
    return dcblk->w;
}                                                                                                                                                                                             

ssize_t SurfaceComposerClient::getDisplayHeight(DisplayID dpy)
{
    if (uint32_t(dpy)>=NUM_DISPLAY_MAX)
        return BAD_VALUE;
    volatile surface_flinger_cblk_t const * cblk = get_cblk();
    volatile display_cblk_t const * dcblk = cblk->displays + dpy;
    return dcblk->h;
}
\end{lstlisting}
       继续追踪到如下代码:
\begin{lstlisting}[language=C++]
surface_flinger_cblk_t const volatile * ComposerService::getControlBlock() {
    return ComposerService::getInstance().mServerCblk;
}

static inline surface_flinger_cblk_t const volatile * get_cblk() {
    return ComposerService::getControlBlock();
}

\end{lstlisting}
        由于调用了ComposerService::getInstance().mServerCblk,所以追踪到ComposerService的声明:
\begin{lstlisting}[language=C++]
class ComposerService : public Singleton<ComposerService>
{
    // these are constants
    sp<ISurfaceComposer> mComposerService;
    sp<IMemoryHeap> mServerCblkMemory;
    surface_flinger_cblk_t volatile* mServerCblk;
    ComposerService();
    friend class Singleton<ComposerService>;
    public:
    static sp<ISurfaceComposer> getComposerService();
    static surface_flinger_cblk_t const volatile * getControlBlock();
};

template <typename TYPE>
class ANDROID_API Singleton
{
    public:
        static TYPE& getInstance() {
            Mutex::Autolock _l(sLock);
            TYPE* instance = sInstance;
            if (instance == 0) {
                instance = new TYPE();
                sInstance = instance;
            }   
            return *instance;
        } 
    ...
}

ComposerService::ComposerService()
    : Singleton<ComposerService>() {
        const String16 name("SurfaceFlinger");
        while (getService(name, &mComposerService) != NO_ERROR) {
            usleep(250000);
        }
        mServerCblkMemory = mComposerService->getCblk();
        mServerCblk = static_cast<surface_flinger_cblk_t volatile *>(                                                                                                                             
                mServerCblkMemory->getBase());
    }
\end{lstlisting}
        从中可以看到在调用ComposerService::getInstance()的时候若没有创建则会创建一个实例
        予以返回,其中mComposerService为binder的client端,我们略过其中的具体机理,直接跳转
        到binder的server端,直接到达如下代码(services/surfaceflinger/SurfaceFlinger.cpp):
\begin{lstlisting}[language=C++]
sp<IMemoryHeap> SurfaceFlinger::getCblk() const
{
    return mServerHeap;
}
\end{lstlisting}
        mServerHeap由如下代码初始化:
\begin{lstlisting}[language=C++]
{
    // initialize the main display
    GraphicPlane& plane(graphicPlane(dpy));
    DisplayHardware* const hw = new DisplayHardware(this, dpy);
    plane.setDisplayHardware(hw);                                                                                                                                                         
} 
mServerHeap = new MemoryHeapBase(4096,
        MemoryHeapBase::READ_ONLY, "SurfaceFlinger read-only heap");
LOGE_IF(mServerHeap==0, "can't create shared memory dealer");

mServerCblk = static_cast<surface_flinger_cblk_t*>(mServerHeap->getBase());
LOGE_IF(mServerCblk==0, "can't get to shared control block's address");

new(mServerCblk) surface_flinger_cblk_t;

// initialize primary screen
// (other display should be initialized in the same manner, but
// asynchronously, as they could come and go. None of this is supported
// yet).
const GraphicPlane& plane(graphicPlane(dpy));
const DisplayHardware& hw = plane.displayHardware();
const uint32_t w = hw.getWidth();
const uint32_t h = hw.getHeight();
const uint32_t f = hw.getFormat();
hw.makeCurrent();

// initialize the shared control block
mServerCblk->connected |= 1<<dpy;
display_cblk_t* dcblk = mServerCblk->displays + dpy;
memset(dcblk, 0, sizeof(display_cblk_t));
dcblk->w            = plane.getWidth();
dcblk->h            = plane.getHeight();
dcblk->format       = f;
dcblk->orientation  = ISurfaceComposer::eOrientationDefault;
dcblk->xdpi         = hw.getDpiX();
dcblk->ydpi         = hw.getDpiY();
dcblk->fps          = hw.getRefreshRate();                                                                                                                                                
dcblk->density      = hw.getDensity();
\end{lstlisting}
        %终于找到了之前对应的\verb|dcblk->w,dcblk->h|。\\
        终于找到了之前对应的\texttt{dcblk->w,dcblk->h}。\\
        其中的plane在前6行进行的初始化,其中GraphicPlane::setDisplayHardware()
        代码如下:
\begin{lstlisting}[language=C++]
int GraphicPlane::getWidth() const {
    return mWidth;
}
int GraphicPlane::getHeight() const {
    return mHeight;
}
void GraphicPlane::setDisplayHardware(DisplayHardware *hw)
{
    mHw = hw;
    ...
    const float w = hw->getWidth();
    const float h = hw->getHeight();
    ...
    if (displayOrientation & ISurfaceComposer::eOrientationSwapMask) {
        mDisplayWidth = h;
        mDisplayHeight = w;
    } else {
        mDisplayWidth = w;
        mDisplayHeight = h;
    }

    setOrientation(ISurfaceComposer::eOrientationDefault);
}
status_t GraphicPlane::setOrientation(int orientation)
{
    const DisplayHardware& hw(displayHardware());
    const float w = mDisplayWidth;
    const float h = mDisplayHeight;
    mWidth = int(w);
    mHeight = int(h);
    ....
    if (orientation & ISurfaceComposer::eOrientationSwapMask) {
        mWidth = int(h);
        mHeight = int(w);
    }
    ....
}
\end{lstlisting}
        从这段代码我们可以看到plane.getWidth()的来历,其实都是通过
        DisplayHardware类获得的,接下来我们分析DisplayHardware是如何
        获得宽和高的数据:
\begin{lstlisting}[language=C++]
DisplayHardware::DisplayHardware(
        const sp<SurfaceFlinger>& flinger,
        uint32_t dpy)
: DisplayHardwareBase(flinger, dpy),
    mFlinger(flinger), mFlags(0), mHwc(0)
{           
    init(dpy);                                                                                                                                                                                
}
void DisplayHardware::init(uint32_t dpy)
{
    //start hdmi and disp service
    DisplayClient disp;
    disp.InitHdmiAndDisp();
    mNativeWindow = new FramebufferNativeWindow();
    ....
    surface = eglCreateWindowSurface(display, config, mNativeWindow.get(), NULL);
    eglQuerySurface(display, surface, EGL_WIDTH,  &mWidth);
    eglQuerySurface(display, surface, EGL_HEIGHT, &mHeight);
}
\end{lstlisting}
        而eglQuerySurface()的代码如下:
\begin{lstlisting}[language=C++]
EGLBoolean eglQuerySurface( EGLDisplay dpy, EGLSurface eglSurface,
        EGLint attribute, EGLint *value)
{
    ...
    switch (attribute) {
        case EGL_CONFIG_ID:
            ret = getConfigAttrib(dpy, surface->config, EGL_CONFIG_ID, value);
            break;
        case EGL_WIDTH:
            *value = surface->getWidth();
            break;
        case EGL_HEIGHT:
            *value = surface->getHeight();
            break;
    ...
}
struct egl_window_surface_v2_t : public egl_surface_t
{   
    ...
    virtual     EGLint      getWidth() const    { return width;  }
    virtual     EGLint      getHeight() const   { return height; }
    ...
}
\end{lstlisting}
        此处的surface实际上为egl\_window\_surface\_v2\_t,所以上面的代码说明了
        eglQuerySurface()查询时的过程。而初始化的这两个成员变量的代码如下:
\begin{lstlisting}[language=C++]
egl_window_surface_v2_t::egl_window_surface_v2_t(EGLDisplay dpy,
        EGLConfig config,
        int32_t depthFormat,
        ANativeWindow* window)
: egl_surface_t(dpy, config, depthFormat), 
    nativeWindow(window),...
{
    ...
    nativeWindow->query(nativeWindow, NATIVE_WINDOW_WIDTH, &width);
    nativeWindow->query(nativeWindow, NATIVE_WINDOW_HEIGHT, &height);
}
\end{lstlisting}
        而此时的nativeWindow其实是之前获得DisplayHardware类对象时
        的FramebufferNativeWindow,
\begin{lstlisting}[language=C++]
    FramebufferNativeWindow::FramebufferNativeWindow() 
: BASE(), fbDev(0), grDev(0), mUpdateOnDemand(false)
{
    hw_module_t const* module;                                                                                                                                                                
    if (hw_get_module(GRALLOC_HARDWARE_MODULE_ID, &module) == 0) {
        err = framebuffer_open(module, &fbDev);
        LOGE_IF(err, "couldn't open framebuffer HAL (%s)", strerror(-err));

        err = gralloc_open(module, &grDev);
        LOGE_IF(err, "couldn't open gralloc HAL (%s)", strerror(-err));
    }
        ....
        ANativeWindow::query = query;
        ...
}

int FramebufferNativeWindow::query(const ANativeWindow* window,
        int what, int* value) 
{
    const FramebufferNativeWindow* self = getSelf(window);
    Mutex::Autolock _l(self->mutex);
    framebuffer_device_t* fb = self->fbDev;
    switch (what) {
        case NATIVE_WINDOW_WIDTH:
            *value = fb->width;
            return NO_ERROR;
        case NATIVE_WINDOW_HEIGHT:
            *value = fb->height;
            return NO_ERROR;
            ...
    }
    *value = 0;
    return BAD_VALUE;
}
\end{lstlisting}
        于是,我们看到了其实query函数使用的宽和高实际是由驱动填写在
        framebuffer\_device\_t里的。
        \item HAL驱动里的详细代码如下:
\begin{lstlisting}[language=C++]
static inline int framebuffer_open(const struct hw_module_t* module,                           
        struct framebuffer_device_t** device) {                                                
    return module->methods->open(module,
            GRALLOC_HARDWARE_FB0, (struct hw_device_t**)device);                               
}
static struct hw_module_methods_t gralloc_module_methods = {
open: gralloc_device_open
};

struct private_module_t HAL_MODULE_INFO_SYM = {
base: {
common: {
    ...
     id: GRALLOC_HARDWARE_MODULE_ID,
     ...
};
nt gralloc_device_open(const hw_module_t* module, const char* name,
        hw_device_t** device)
{
    int status = -EINVAL;
    if (!strcmp(name, GRALLOC_HARDWARE_GPU0)) {
        ...
    } else {
        status = fb_device_open(module, name, device);
    }
    return status;
}

int fb_device_open(hw_module_t const* module, const char* name,
        hw_device_t** device)
{
    if (!strcmp(name, GRALLOC_HARDWARE_FB0)) {
        ...
        /* initialize our state here */
        fb_context_t *dev = (fb_context_t*)malloc(sizeof(*dev));

        ...        
        private_module_t* m = (private_module_t*)module;
        status = mapFrameBuffer(m);
        if (status >= 0) {
            ....
            const_cast<uint32_t&>(dev->device.width) = m->info.xres;
            const_cast<uint32_t&>(dev->device.height) = m->info.yres;
            const_cast<int&>(dev->device.stride) = stride;
            const_cast<int&>(dev->device.format) = format;
            const_cast<float&>(dev->device.xdpi) = m->xdpi;
            const_cast<float&>(dev->device.ydpi) = m->ydpi;
            const_cast<float&>(dev->device.fps) = m->fps;
            ...
            *device = &dev->device.common;
        }
    }
}

int mapFrameBufferLocked(struct private_module_t* module)
{
    // already initialized...
    if (module->framebuffer) {
        return 0;
    }

    char const * const device_template[] = {
        "/dev/graphics/fb%u",
        "/dev/fb%u",                                                                                                                                                                      
        0 };

    struct fb_fix_screeninfo finfo;
    if (ioctl(fd, FBIOGET_FSCREENINFO, &finfo) == -1)
        return -errno;

    struct fb_var_screeninfo info;
    if (ioctl(fd, FBIOGET_VSCREENINFO, &info) == -1)
        return -errno;
    ...
    if (int(info.width) <= 0 || int(info.height) <= 0) {
        // the driver doesn't return that information
        // default to 160 dpi
        info.width  = ((info.xres * 25.4f)/160.0f + 0.5f);
        info.height = ((info.yres * 25.4f)/160.0f + 0.5f);
    }

    float xdpi = (info.xres * 25.4f) / info.width;
    float ydpi = (info.yres * 25.4f) / info.height;
    float fps  = refreshRate / 1000.0f;
    module->flags = flags;
    module->info = info;
    module->finfo = finfo;
    module->xdpi = xdpi;
    module->ydpi = ydpi;
    module->fps = fps;
}
\end{lstlisting}
        由于我们的驱动中没有返回info.width和info.height,所以我们的DPI默认为
        160。\\
        至此,我们把判断显示PhoneStatusBar还是TabletStatusBar的代码全部分析完毕。
    \end{enumerate}


\section{总结}
    \begin{enumerate}
    \item 当在驱动里将分辨率的宽和高两个值的最小值设置为大于等于600的值时,
        Android将自动使用TabletStatusBar,反之则使用PhoneStatusBar。

    \item 当需要通过软件方式修改DPI时,可以设置ro.sf.lcd\_density属性
    \item 当不需要启动SystemUI时,可以在SystemServer启动代码里将startSystemUi()注释掉
    \end{enumerate}



\section{强行使用PhoneStatusBar}
在SystemUIService.onCreate()中将以下代码
\begin{lstlisting}[language=JAVA]
SERVICES[0] = wm.canStatusBarHide()
    ? R.string.config_statusBarComponent
    : R.string.config_systemBarComponent;
}
\end{lstlisting}
强行改为:
\begin{lstlisting}[language=JAVA]
SERVICES[0] = R.string.config_statusBarComponent;
}
\end{lstlisting}
    同时还要注意的是无论在PhoneStatusBar还是TabletStatusBar中,加载
    资源文件时都是加载的R.layout.status\_bar,如下:
\begin{lstlisting}[language=JAVA]
PhoneStatusBarView sb = (PhoneStatusBarView)View.inflate(context, R.layout.status_bar, null);
\end{lstlisting}
\begin{lstlisting}[language=JAVA]
final TabletStatusBarView sb = (TabletStatusBarView)View.inflate(context, R.layout.status_bar, null);
\end{lstlisting}
    所以在不同的显示分辨率时要注意确认res/layout文件夹以及layout-sw600dp[mdpi/hdpi/xhdpi]里的
    statusbar.xml里的内容要与代码一致。
\end{CJK*} %有始有終
\end{document}
